
%% bare_conf.tex
%% V1.3
%% 2007/01/11
%% by Michael Shell
%% See:
%% http://www.michaelshell.org/
%% for current contact information.
%%
%% This is a skeleton file demonstrating the use of IEEEtran.cls
%% (requires IEEEtran.cls version 1.7 or later) with an IEEE conference paper.
%%
%% Support sites:
%% http://www.michaelshell.org/tex/ieeetran/
%% http://www.ctan.org/tex-archive/macros/latex/contrib/IEEEtran/
%% and
%% http://www.ieee.org/

%%*************************************************************************
%% Legal Notice:
%% This code is offered as-is without any warranty either expressed or
%% implied; without even the implied warranty of MERCHANTABILITY or
%% FITNESS FOR A PARTICULAR PURPOSE! 
%% User assumes all risk.
%% In no event shall IEEE or any contributor to this code be liable for
%% any damages or losses, including, but not limited to, incidental,
%% consequential, or any other damages, resulting from the use or misuse
%% of any information contained here.
%%
%% All comments are the opinions of their respective authors and are not
%% necessarily endorsed by the IEEE.
%%
%% This work is distributed under the LaTeX Project Public License (LPPL)
%% ( http://www.latex-project.org/ ) version 1.3, and may be freely used,
%% distributed and modified. A copy of the LPPL, version 1.3, is included
%% in the base LaTeX documentation of all distributions of LaTeX released
%% 2003/12/01 or later.
%% Retain all contribution notices and credits.
%% ** Modified files should be clearly indicated as such, including  **
%% ** renaming them and changing author support contact information. **
%%
%% File list of work: IEEEtran.cls, IEEEtran_HOWTO.pdf, bare_adv.tex,
%%                    bare_conf.tex, bare_jrnl.tex, bare_jrnl_compsoc.tex
%%*************************************************************************

% *** Authors should verify (and, if needed, correct) their LaTeX system  ***
% *** with the testflow diagnostic prior to trusting their LaTeX platform ***
% *** with production work. IEEE's font choices can trigger bugs that do  ***
% *** not appear when using other class files.                            ***
% The testflow support page is at:
% http://www.michaelshell.org/tex/testflow/



% Note that the a4paper option is mainly intended so that authors in
% countries using A4 can easily print to A4 and see how their papers will
% look in print - the typesetting of the document will not typically be
% affected with changes in paper size (but the bottom and side margins will).
% Use the testflow package mentioned above to verify correct handling of
% both paper sizes by the user's LaTeX system.
%
% Also note that the "draftcls" or "draftclsnofoot", not "draft", option
% should be used if it is desired that the figures are to be displayed in
% draft mode.
%
\documentclass[conference]{IEEEtran}
% Add the compsoc option for Computer Society conferences.
%
% If IEEEtran.cls has not been installed into the LaTeX system files,
% manually specify the path to it like:
% \documentclass[conference]{../sty/IEEEtran}





% Some very useful LaTeX packages include:
% (uncomment the ones you want to load)


% *** MISC UTILITY PACKAGES ***
%
%\usepackage{ifpdf}
% Heiko Oberdiek's ifpdf.sty is very useful if you need conditional
% compilation based on whether the output is pdf or dvi.
% usage:
% \ifpdf
%   % pdf code
% \else
%   % dvi code
% \fi
% The latest version of ifpdf.sty can be obtained from:
% http://www.ctan.org/tex-archive/macros/latex/contrib/oberdiek/
% Also, note that IEEEtran.cls V1.7 and later provides a builtin
% \ifCLASSINFOpdf conditional that works the same way.
% When switching from latex to pdflatex and vice-versa, the compiler may
% have to be run twice to clear warning/error messages.






% *** CITATION PACKAGES ***
%
%\usepackage{cite}
% cite.sty was written by Donald Arseneau
% V1.6 and later of IEEEtran pre-defines the format of the cite.sty package
% \cite{} output to follow that of IEEE. Loading the cite package will
% result in citation numbers being automatically sorted and properly
% "compressed/ranged". e.g., [1], [9], [2], [7], [5], [6] without using
% cite.sty will become [1], [2], [5]--[7], [9] using cite.sty. cite.sty's
% \cite will automatically add leading space, if needed. Use cite.sty's
% noadjust option (cite.sty V3.8 and later) if you want to turn this off.
% cite.sty is already installed on most LaTeX systems. Be sure and use
% version 4.0 (2003-05-27) and later if using hyperref.sty. cite.sty does
% not currently provide for hyperlinked citations.
% The latest version can be obtained at:
% http://www.ctan.org/tex-archive/macros/latex/contrib/cite/
% The documentation is contained in the cite.sty file itself.






% *** GRAPHICS RELATED PACKAGES ***
%
\ifCLASSINFOpdf
  % \usepackage[pdftex]{graphicx}
  % declare the path(s) where your graphic files are
  % \graphicspath{{../pdf/}{../jpeg/}}
  % and their extensions so you won't have to specify these with
  % every instance of \includegraphics
  % \DeclareGraphicsExtensions{.pdf,.jpeg,.png}
\else
  % or other class option (dvipsone, dvipdf, if not using dvips). graphicx
  % will default to the driver specified in the system graphics.cfg if no
  % driver is specified.
  % \usepackage[dvips]{graphicx}
  % declare the path(s) where your graphic files are
  % \graphicspath{{../eps/}}
  % and their extensions so you won't have to specify these with
  % every instance of \includegraphics
  % \DeclareGraphicsExtensions{.eps}
\fi
% graphicx was written by David Carlisle and Sebastian Rahtz. It is
% required if you want graphics, photos, etc. graphicx.sty is already
% installed on most LaTeX systems. The latest version and documentation can
% be obtained at: 
% http://www.ctan.org/tex-archive/macros/latex/required/graphics/
% Another good source of documentation is "Using Imported Graphics in
% LaTeX2e" by Keith Reckdahl which can be found as epslatex.ps or
% epslatex.pdf at: http://www.ctan.org/tex-archive/info/
%
% latex, and pdflatex in dvi mode, support graphics in encapsulated
% postscript (.eps) format. pdflatex in pdf mode supports graphics
% in .pdf, .jpeg, .png and .mps (metapost) formats. Users should ensure
% that all non-photo figures use a vector format (.eps, .pdf, .mps) and
% not a bitmapped formats (.jpeg, .png). IEEE frowns on bitmapped formats
% which can result in "jaggedy"/blurry rendering of lines and letters as
% well as large increases in file sizes.
%
% You can find documentation about the pdfTeX application at:
% http://www.tug.org/applications/pdftex





% *** MATH PACKAGES ***
%
%\usepackage[cmex10]{amsmath}
% A popular package from the American Mathematical Society that provides
% many useful and powerful commands for dealing with mathematics. If using
% it, be sure to load this package with the cmex10 option to ensure that
% only type 1 fonts will utilized at all point sizes. Without this option,
% it is possible that some math symbols, particularly those within
% footnotes, will be rendered in bitmap form which will result in a
% document that can not be IEEE Xplore compliant!
%
% Also, note that the amsmath package sets \interdisplaylinepenalty to 10000
% thus preventing page breaks from occurring within multiline equations. Use:
%\interdisplaylinepenalty=2500
% after loading amsmath to restore such page breaks as IEEEtran.cls normally
% does. amsmath.sty is already installed on most LaTeX systems. The latest
% version and documentation can be obtained at:
% http://www.ctan.org/tex-archive/macros/latex/required/amslatex/math/





% *** SPECIALIZED LIST PACKAGES ***
%
%\usepackage{algorithmic}
% algorithmic.sty was written by Peter Williams and Rogerio Brito.
% This package provides an algorithmic environment fo describing algorithms.
% You can use the algorithmic environment in-text or within a figure
% environment to provide for a floating algorithm. Do NOT use the algorithm
% floating environment provided by algorithm.sty (by the same authors) or
% algorithm2e.sty (by Christophe Fiorio) as IEEE does not use dedicated
% algorithm float types and packages that provide these will not provide
% correct IEEE style captions. The latest version and documentation of
% algorithmic.sty can be obtained at:
% http://www.ctan.org/tex-archive/macros/latex/contrib/algorithms/
% There is also a support site at:
% http://algorithms.berlios.de/index.html
% Also of interest may be the (relatively newer and more customizable)
% algorithmicx.sty package by Szasz Janos:
% http://www.ctan.org/tex-archive/macros/latex/contrib/algorithmicx/




% *** ALIGNMENT PACKAGES ***
%
%\usepackage{array}
% Frank Mittelbach's and David Carlisle's array.sty patches and improves
% the standard LaTeX2e array and tabular environments to provide better
% appearance and additional user controls. As the default LaTeX2e table
% generation code is lacking to the point of almost being broken with
% respect to the quality of the end results, all users are strongly
% advised to use an enhanced (at the very least that provided by array.sty)
% set of table tools. array.sty is already installed on most systems. The
% latest version and documentation can be obtained at:
% http://www.ctan.org/tex-archive/macros/latex/required/tools/


%\usepackage{mdwmath}
%\usepackage{mdwtab}
% Also highly recommended is Mark Wooding's extremely powerful MDW tools,
% especially mdwmath.sty and mdwtab.sty which are used to format equations
% and tables, respectively. The MDWtools set is already installed on most
% LaTeX systems. The lastest version and documentation is available at:
% http://www.ctan.org/tex-archive/macros/latex/contrib/mdwtools/


% IEEEtran contains the IEEEeqnarray family of commands that can be used to
% generate multiline equations as well as matrices, tables, etc., of high
% quality.


%\usepackage{eqparbox}
% Also of notable interest is Scott Pakin's eqparbox package for creating
% (automatically sized) equal width boxes - aka "natural width parboxes".
% Available at:
% http://www.ctan.org/tex-archive/macros/latex/contrib/eqparbox/





% *** SUBFIGURE PACKAGES ***
%\usepackage[tight,footnotesize]{subfigure}
% subfigure.sty was written by Steven Douglas Cochran. This package makes it
% easy to put subfigures in your figures. e.g., "Figure 1a and 1b". For IEEE
% work, it is a good idea to load it with the tight package option to reduce
% the amount of white space around the subfigures. subfigure.sty is already
% installed on most LaTeX systems. The latest version and documentation can
% be obtained at:
% http://www.ctan.org/tex-archive/obsolete/macros/latex/contrib/subfigure/
% subfigure.sty has been superceeded by subfig.sty.



%\usepackage[caption=false]{caption}
%\usepackage[font=footnotesize]{subfig}
% subfig.sty, also written by Steven Douglas Cochran, is the modern
% replacement for subfigure.sty. However, subfig.sty requires and
% automatically loads Axel Sommerfeldt's caption.sty which will override
% IEEEtran.cls handling of captions and this will result in nonIEEE style
% figure/table captions. To prevent this problem, be sure and preload
% caption.sty with its "caption=false" package option. This is will preserve
% IEEEtran.cls handing of captions. Version 1.3 (2005/06/28) and later 
% (recommended due to many improvements over 1.2) of subfig.sty supports
% the caption=false option directly:
%\usepackage[caption=false,font=footnotesize]{subfig}
%
% The latest version and documentation can be obtained at:
% http://www.ctan.org/tex-archive/macros/latex/contrib/subfig/
% The latest version and documentation of caption.sty can be obtained at:
% http://www.ctan.org/tex-archive/macros/latex/contrib/caption/




% *** FLOAT PACKAGES ***
%
%\usepackage{fixltx2e}
% fixltx2e, the successor to the earlier fix2col.sty, was written by
% Frank Mittelbach and David Carlisle. This package corrects a few problems
% in the LaTeX2e kernel, the most notable of which is that in current
% LaTeX2e releases, the ordering of single and double column floats is not
% guaranteed to be preserved. Thus, an unpatched LaTeX2e can allow a
% single column figure to be placed prior to an earlier double column
% figure. The latest version and documentation can be found at:
% http://www.ctan.org/tex-archive/macros/latex/base/



%\usepackage{stfloats}
% stfloats.sty was written by Sigitas Tolusis. This package gives LaTeX2e
% the ability to do double column floats at the bottom of the page as well
% as the top. (e.g., "\begin{figure*}[!b]" is not normally possible in
% LaTeX2e). It also provides a command:
%\fnbelowfloat
% to enable the placement of footnotes below bottom floats (the standard
% LaTeX2e kernel puts them above bottom floats). This is an invasive package
% which rewrites many portions of the LaTeX2e float routines. It may not work
% with other packages that modify the LaTeX2e float routines. The latest
% version and documentation can be obtained at:
% http://www.ctan.org/tex-archive/macros/latex/contrib/sttools/
% Documentation is contained in the stfloats.sty comments as well as in the
% presfull.pdf file. Do not use the stfloats baselinefloat ability as IEEE
% does not allow \baselineskip to stretch. Authors submitting work to the
% IEEE should note that IEEE rarely uses double column equations and
% that authors should try to avoid such use. Do not be tempted to use the
% cuted.sty or midfloat.sty packages (also by Sigitas Tolusis) as IEEE does
% not format its papers in such ways.





% *** PDF, URL AND HYPERLINK PACKAGES ***
%
\usepackage{url}
% url.sty was written by Donald Arseneau. It provides better support for
% handling and breaking URLs. url.sty is already installed on most LaTeX
% systems. The latest version can be obtained at:
% http://www.ctan.org/tex-archive/macros/latex/contrib/misc/
% Read the url.sty source comments for usage information. Basically,
% \url{my_url_here}.





% *** Do not adjust lengths that control margins, column widths, etc. ***
% *** Do not use packages that alter fonts (such as pslatex).         ***
% There should be no need to do such things with IEEEtran.cls V1.6 and later.
% (Unless specifically asked to do so by the journal or conference you plan
% to submit to, of course. )


% correct bad hyphenation here
\hyphenation{op-tical net-works semi-conduc-tor}


\begin{document}
%
% paper title
% can use linebreaks \\ within to get better formatting as desired
\title{Tracking of a Fleet of Paparazzi Drones}


% author names and affiliations
% use a multiple column layout for up to three different
% affiliations
\author{\IEEEauthorblockN{Isabelle Santos}
\IEEEauthorblockA{A20325602\\isantos1@hawk.iit.edu}
\and
\IEEEauthorblockN{Dheeraj Prajwal Bojja Venkata}
\IEEEauthorblockA{A20329437\\prajdheeraj@gmail.com}}

% conference papers do not typically use \thanks and this command
% is locked out in conference mode. If really needed, such as for
% the acknowledgment of grants, issue a \IEEEoverridecommandlockouts
% after \documentclass

% for over three affiliations, or if they all won't fit within the width
% of the page, use this alternative format:
% 
%\author{\IEEEauthorblockN{Michael Shell\IEEEauthorrefmark{1},
%Homer Simpson\IEEEauthorrefmark{2},
%James Kirk\IEEEauthorrefmark{3}, 
%Montgomery Scott\IEEEauthorrefmark{3} and
%Eldon Tyrell\IEEEauthorrefmark{4}}
%\IEEEauthorblockA{\IEEEauthorrefmark{1}School of Electrical and Computer Engineering\\
%Georgia Institute of Technology,
%Atlanta, Georgia 30332--0250\\ Email: see http://www.michaelshell.org/contact.html}
%\IEEEauthorblockA{\IEEEauthorrefmark{2}Twentieth Century Fox, Springfield, USA\\
%Email: homer@thesimpsons.com}
%\IEEEauthorblockA{\IEEEauthorrefmark{3}Starfleet Academy, San Francisco, California 96678-2391\\
%Telephone: (800) 555--1212, Fax: (888) 555--1212}
%\IEEEauthorblockA{\IEEEauthorrefmark{4}Tyrell Inc., 123 Replicant Street, Los Angeles, California 90210--4321}}




% use for special paper notices
%\IEEEspecialpapernotice{(Invited Paper)}




% make the title area
\maketitle


\begin{abstract}
%\boldmath
The safety pilot of a drone can almost only rely on their eyes to determine the state of a drone they are monitoring. In the context for instance of a search and rescue mission, many drones could be involved, and the pilot may lose sight of the drone they are monitoring, and some drones may go beyond the range of the safety pilot's remote control. This project proposal envisions and describes the design and implementation of a smartphone app to monitor and control a fleet of drones with an interface similar to that of GoogleSky.
\end{abstract}
% IEEEtran.cls defaults to using nonbold math in the Abstract.
% This preserves the distinction between vectors and scalars. However,
% if the conference you are submitting to favors bold math in the abstract,
% then you can use LaTeX's standard command \boldmath at the very start
% of the abstract to achieve this. Many IEEE journals/conferences frown on
% math in the abstract anyway.

% no keywords




% For peer review papers, you can put extra information on the cover
% page as needed:
% \ifCLASSOPTIONpeerreview
% \begin{center} \bfseries EDICS Category: 3-BBND \end{center}
% \fi
%
% For peerreview papers, this IEEEtran command inserts a page break and
% creates the second title. It will be ignored for other modes.
\IEEEpeerreviewmaketitle



% An example of a floating figure using the graphicx package.
% Note that \label must occur AFTER (or within) \caption.
% For figures, \caption should occur after the \includegraphics.
% Note that IEEEtran v1.7 and later has special internal code that
% is designed to preserve the operation of \label within \caption
% even when the captionsoff option is in effect. However, because
% of issues like this, it may be the safest practice to put all your
% \label just after \caption rather than within \caption{}.
%
% Reminder: the "draftcls" or "draftclsnofoot", not "draft", class
% option should be used if it is desired that the figures are to be
% displayed while in draft mode.
%
%\begin{figure}[!t]
%\centering
%\includegraphics[width=2.5in]{myfigure}
% where an .eps filename suffix will be assumed under latex, 
% and a .pdf suffix will be assumed for pdflatex; or what has been declared
% via \DeclareGraphicsExtensions.
%\caption{Simulation Results}
%\label{fig_sim}
%\end{figure}

% Note that IEEE typically puts floats only at the top, even when this
% results in a large percentage of a column being occupied by floats.


% An example of a double column floating figure using two subfigures.
% (The subfig.sty package must be loaded for this to work.)
% The subfigure \label commands are set within each subfloat command, the
% \label for the overall figure must come after \caption.
% \hfil must be used as a separator to get equal spacing.
% The subfigure.sty package works much the same way, except \subfigure is
% used instead of \subfloat.
%
%\begin{figure*}[!t]
%\centerline{\subfloat[Case I]\includegraphics[width=2.5in]{subfigcase1}%
%\label{fig_first_case}}
%\hfil
%\subfloat[Case II]{\includegraphics[width=2.5in]{subfigcase2}%
%\label{fig_second_case}}}
%\caption{Simulation results}
%\label{fig_sim}
%\end{figure*}
%
% Note that often IEEE papers with subfigures do not employ subfigure
% captions (using the optional argument to \subfloat), but instead will
% reference/describe all of them (a), (b), etc., within the main caption.


% An example of a floating table. Note that, for IEEE style tables, the 
% \caption command should come BEFORE the table. Table text will default to
% \footnotesize as IEEE normally uses this smaller font for tables.
% The \label must come after \caption as always.
%
%\begin{table}[!t]
%% increase table row spacing, adjust to taste
%\renewcommand{\arraystretch}{1.3}
% if using array.sty, it might be a good idea to tweak the value of
% \extrarowheight as needed to properly center the text within the cells
%\caption{An Example of a Table}
%\label{table_example}
%\centering
%% Some packages, such as MDW tools, offer better commands for making tables
%% than the plain LaTeX2e tabular which is used here.
%\begin{tabular}{|c||c|}
%\hline
%One & Two\\
%\hline
%Three & Four\\
%\hline
%\end{tabular}
%\end{table}


% Note that IEEE does not put floats in the very first column - or typically
% anywhere on the first page for that matter. Also, in-text middle ("here")
% positioning is not used. Most IEEE journals/conferences use top floats
% exclusively. Note that, LaTeX2e, unlike IEEE journals/conferences, places
% footnotes above bottom floats. This can be corrected via the \fnbelowfloat
% command of the stfloats package.




\section{Introduction}
\subsection{Context}
Paparazzi is a free and open-source software project \cite{Paparazzi} comprising the hardware and software necessary deploy a drone autopilot system. Paparazzi has been used by universities, private companies and drone enthusiasts. 

Paparazzi is articulated around a ground control system (GCS) and the actual drone. The drone is composed of modules for receiving orders from the ground station, modules for sending status reports to the ground station, and flight control systems and actuators. The GCS is generally a computer running the Paparazzi software, with an antenna to emit and receive messages to and from the drone and a standard RC transmitter. 

During the flight of a drone, there are two pilots:
\begin{itemize}
\item the GCS operator inputs flight parameters and sees the flight data in real-time. He follows the trajectory of the drone on a map;
\item the safety pilot observes the drone and holds ready to take command of in case anything unexpected happens. He follows the trajectory of the drone in the sky. 
\end{itemize}

\subsubsection{Limitations of the current system} 
\paragraph{Quality of information transmission} The safety pilot communicates with the GCS operator (for instance by talkie-walkie) in order to get the drone flight data. The transmission can be of poor quality, resulting in incomprehensions. If the communication between both operators is cut off, the flight becomes dangerous.

\paragraph{Information reception time} There is a high latency between the moment when the drone sends out information and the moment when the safety pilot receives the information.

\paragraph{Drone locating} The safety pilot has no way of easily finding the position of a drone. If the safety pilot loses sight of a drone, they must go to the GCS, look at the map, and try to spot the drone. 

\paragraph{Drone identification} The safety pilot can not identify a drone that they see without prior knowledge. Two drones may look very similar, and distinguishing them by sight from a distance may be impossible. 

\paragraph{Fleet management} The safety pilot usually has only a standard RC model remote control, so typically a pilot only follows one drone at a time. This limits the feasibility of a fleet of drones. 

\paragraph{Flight range} If a drone flies out of the range of the GCS and the safety pilot's remote control, it can continue to follow its flight plan. However, it is no longer possible to send orders to that drone, nor is it possible to retrieve information. If something unexpected happens when the drone is out of range, the drone could very well get lost, injure people or damage property.


\subsection{Presentation}
This project would consist in creating a smartphone app for Android that would interface with the Paparazzi system to reduce the workload for the drone safety pilot, enable the management of more than one drone at a time, and enable safe flight beyond the range of the GCS under certain conditions. Since the app can be used on a smartphone, it can be carried around easily by the safety pilot for field use. 

This app enables the pilot to visualize information concerning all registered drones. This app may also be used to monitor and control a fleet of drones. In particular, the pilot should be able to access all relevant information concerning a drone, as well as localize or identify a drone in the sky. The app also enables the pilot to send orders to a drone from the smartphone. 

It should also be possible for the drone to send a message to the safety pilot. The smartphone then indicates the position or the direction of the drone. Such messages can be relative to the drone status (e.g. low battery), or the drone environment if said drone is outfitted with sensors. 

The final part of this project is to create an Vehicular ad-hoc Network (VANET) within the flock of drones and test their connectivity and communication. In other terms, there will be a Control Center (laptop/smartphone) that will monitor one of the drones which is designated as a Road Side Unit (RSU) and will communicate with the other drones which are designated as On Board Units (OBUs or nodes). 

The communication system will be based off of current advancements in collective behaviour and may be inspired by existing communication systems such as ADS-B \cite{faaAdsbServices, faaAdsbArchitecture} which allows air-to-air communications as well as air-to-infrastructure communications. 
The work will be based on existing the existing Paparazzi framework, and will use associated software such as the mission status report module, the real-time plotter (RTPlotter), and the server application (AppServer) \cite{Paparazzi}.

Possible applications for this project include to monitor traffic as to avoid traffic jams where an accident may have occurred, or search and rescue missions where deploying people could be dangerous. 


\subsection{Related Work}
Following an aircraft in flight has been the subject of a number of studies over the years. 

Many smartphone based solutions such as FlightAware \cite{FlightAware} or Flight Radar \cite{FlightRadar24} enable the user to know the position, flight status and other information concerning most commercial flights in the world, and follow flights live. 

Solutions also exist for controlling drones with a smartphone. For instance ARDrone Flight is the companion app to the Parrot ARDrone for controlling the drone and recording videos \cite{ARDroneFlight,ARDrone}. However, it does not provide visual tracking using the phone's camera, and it is not open-source. The PPRZonDroid \cite{PPRZonDroid} is an application that you can use to control Paparazzi aircraft with your android device. However, it only replaces the GCS and the user can only follow the trajectory of a drone on a map. 

GoogleSky \cite{GoogleSky} features tracking. It is a Google app that shows  the celestial sphere on the smartphone screen by linking the screen display with the screen orientation. In the case of this project, the stars of GoogleSky would be replaced by a fleet of micro-drones. 

Recent studies have also been done on the subject of swarming of ground robots or aerial drones \cite{Vasarhelyi14}, collective behavior, and flying ad-hoc networks (FANET) \cite{VeyPRG14, RoyerPG13}. Many applications for such FANETs are envisionned such as meteorological studies  \cite{hattenberger13}, thermal detection for hang-gliders \cite{eckert2013flying} or search and rescue missions \cite{Cubber13}.

Studies in \cite{Abrassart14, bouachir14} shows how the ad hoc network in the context of unmanned aerial systems (UAS) is a promising solution.

The apps that currently exist do not integrate drone control, drone monitoring and tracking of multiple drones in a single app. Therefore, even though this project is inspired by existing applications, it is unlike each of those applications.


\section{Implementation}
The app is divided into four layers
\begin{itemize}
\item the user interface,
\item the model,
\item the network,
\item the collective behaviour.
\end{itemize}

\subsection{User interface}
This layer consists of what the safety pilots sees and interacts with through their smartphone.

The app has four distinct screens:
\begin{itemize}
\item Flight listing: the list of all drones currently tracked by the app
\item Flight data: all information relevant to a selected drone
\item Tracking: symbolic representation of the drones over the image taken by the phone camera
\item Preferences: network connection parameters, selection of warnings to be displayed,...
\end{itemize}

This layer includes the threads necessary to switch from one screen to another.

\subsection{Model and controller}
This layer includes all the information specific to each drone and the smartphone device. The main elements of the model are the UAV and the Fleet.

\subsubsection{UAV flight data}
This is the most important part of the model, since it concerns critical data. The safety pilot makes decisions concerning the drones based on this data. For each drone, the data that we consider is the following:
\begin{itemize}
\item Name, colour and id
\item Vertical speed
\item Ground speed: this is the speed of the drone relatively to the local Earth frame
\item True air speed: this is the speed of the drone relatively to the body of air it is flying through
\item Orientation characterised by the roll, pitch and heading of the drone
\item Coordinates of longitude and latitude
\item Height: this is the distance from the drone to the ground directly under it
\item Altitude: this is the vertical position of the drone above the mean sea level
\end{itemize}

The information concerning each drone is transmitted by the parser of the Networks layer, and written to the screen of the User Interface layer on a regular basis. 

\subsubsection{Fleet}
A fleet contains several drones and is controlled by a surveillance device.

The Fleet class is a doorway between the view and the model. Fleet manages the list of drones which can change dynamically.

\subsubsection{Device data}
The data regarding the device includes the orientation and location of the phone. This is used in combination with the location of the drone to provide the tracking feature. The following data regarding the device is considered:
\begin{itemize}
\item Orientation characterised by the roll, pitch and azimuth of the device
\item Coordinates of altitude, longitude and latitude
\item Cone of visibility of the camera
\end{itemize}

The LocationManager and SensorManager classes of the android library are used to provide this information.

\subsubsection{User preferences}
The preferences that may be set by the user are of several sorts:
\begin{itemize}
\item Network preferences: the user may need to modify the IP address or connexion ports
\item Warning preferences: the user may want to activate/deactivate certain types of warnings based on the type of mission performed
\item ...
\end{itemize}

\subsection{Networks}
This layer includes a parser for converting messages received into data usable by the other layers for the device-drone communications.

The parser does the following tasks:
\begin{itemize}
\item establish a connexion;
\item receive messages;
\item acknowledge the reception of messages;
\item process messages;
\item format the collected data;
\item transmit the collected data to the other modules.
\end{itemize}

A dedicated thread establishes the connexion with the appServer. Another thread acknowledges the reception of TCP messages.

\subsection{Collective behaviour}
This layer includes all the rules for the drones to interact amongst themselves.

If two drones are within communication range, they can compute the distance separating them by measuring the time to transmit messages.


\section{Tests}
Only the app and the ad hoc network between drones are being tested. It is assumed for this project that Paparazzi and the smartphone are free of bugs. 

The features to be tested include the display of the app on the smartphone, the conformity of the information displayed to the safety pilot, and the communications between drones. 

The Paparazzi software can be used to simulate the flight of drones. Information can be sent to and from the simulated drones in real-time using the network created by AppServer, the server launched by Paparazzi to obtain information from the drones. It is then possible to connect a smartphone to this network and obtain the simulated flight data and send instructions to the simulated drone. 

%For each test involving the simulation of drones, the Paparazzi software must be launched and configured, and a wifi network must be created to connect the Paparazzi AppServer and the smartphone. This includes getting the IP addresses for both in order to configure the parameters of the app. 

\subsection{Tests in a simulated environment}

\subsection{Tests in a real environment}


\section{Conclusion and future work}
Conclusions will be drawn from this project, future work will be envisioned. 





% use section* for acknowledgement
\section*{Acknowledgment}
The authors would like to thank Xiang-Yang Li.




% trigger a \newpage just before the given reference
% number - used to balance the columns on the last page
% adjust value as needed - may need to be readjusted if
% the document is modified later
%\IEEEtriggeratref{8}
% The "triggered" command can be changed if desired:
%\IEEEtriggercmd{\enlargethispage{-5in}}

% references section

% can use a bibliography generated by BibTeX as a .bbl file
% BibTeX documentation can be easily obtained at:
% http://www.ctan.org/tex-archive/biblio/bibtex/contrib/doc/
% The IEEEtran BibTeX style support page is at:
% http://www.michaelshell.org/tex/ieeetran/bibtex/
%\bibliographystyle{IEEEtran}
% argument is your BibTeX string definitions and bibliography database(s)
%\bibliography{IEEEabrv,../bib/paper}
%
% <OR> manually copy in the resultant .bbl file
% set second argument of \begin to the number of references
% (used to reserve space for the reference number labels box)
%\begin{thebibliography}{1}
%
%\bibitem{IEEEhowto:kopka}
%H.~Kopka and P.~W. Daly, \emph{A Guide to \LaTeX}, 3rd~ed.\hskip 1em plus
%  0.5em minus 0.4em\relax Harlow, England: Addison-Wesley, 1999.
%
%\end{thebibliography}

\bibliographystyle{IEEEtran}
\bibliography{ProjectRefs}


% that's all folks
\end{document}


